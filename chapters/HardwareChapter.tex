\Chapter{Hardware}\label{chapter:hardwareChapter}

\Section{Introduction}\\

A typical CAVE installation has four major sources of hardware: projectors, screens, a tracking system, and a structure to mount everything. High resolution CAVEs can contain multiple displays per screen and become difficult to setup.

--- FIX ME ---

%\ref{ref1}.

\Section{Projectors}\label{sec:hwProjectorsSection}\\

The display of computer generated content is dominated by projectors and monitors. Projectors use a lamp and lens to project light onto a surface. Monitors utilize a two dimensional array of light cells to present an image to the user.

You can think of these technologies on a loose spectrum, similar to figure 1. On the left end, a large or far image is required, this is ideal for projectors because the resultant image size is a factor of throw distance. As you move closer to the monitors, a close or small image is preferred. This spectrum is loose because you can apply monitor technology that is ideal for a projector (and vice versa), there is an increase in cost and innovation to make it happen.

The ideal CAVE system would be powered by monitors. They are superior in terms of pixel density and resolution, fast response times, physical size, and ease of calibration. A disadvantage is their bezel will cause some issues with rendering and make content like text hard to read. When we increase the pixel density and resolution, there is a increase in power and computing requirements. A cluster of PCs will be required to render and this will drive costs up. Due to this, affordable CAVE systems prefer projector setups.

Although projection based CAVEs suffer from pixel resolution and brightness, we see a savings in computing and cost. It is not necessary for a CAVE to have multiple projectors per wall, although high-end CAVEs may include it to help bridge the gap in performance between the two techniques.

In this project, we decided with using a single projector per wall in order to minimize the complexity and cost of the system. We created a few requirements to filter when we performed a search.

\begin{center}
	\begin{tabular}{|c|c|}
		\hline 
		1 & 1920x1080 Resolution \\ 
		\hline 
		2 & Capable of 3D stereoscopy \\ 
		\hline 
		3 & Ultra Short Throw \\ 
		\hline 
		4 & Less than \$2000 MSRP \\ 
		\hline 
	\end{tabular} 
\end{center}

An EBU Technical Report from 2012 states that 1080p is technically mature and almost all new flat panel displays use FullHD. By 2012, some 4K displays have entered the high-end sector of the market. \cite{ebuHDTVRef} In 2019, 4K flat panels are becoming commonplace and FullHD displays are ubiquitous. Therefore, we make FullHD a requirement for the resolution of the projector.

Reading the history of affordable IPTs, multiple projectors per wall is a common theme to enable stereoscopic imagery. Each projector would represent an eye and some technique of shuttering is used to block light. Stereoscopic alignment and portability are special concerns for this technique. Multiple projectors are necessary because 


\clearpage
