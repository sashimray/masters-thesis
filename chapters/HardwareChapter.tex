\Chapter{Hardware}\label{chapter:hardwareChapter}

\Section{Introduction}\\

A typical CAVE installation has four major sources of hardware: projectors, screens, a tracking system, and a structure to mount everything. High resolution CAVEs can contain multiple displays per screen and become difficult to setup.

--- FIX ME ---

%\ref{ref1}.

\Section{Projectors}\label{sec:hwProjectorsSection}\\
\begin{center}
	\textcolor{OliveGreen}{Purpose: What was the process of choosing this projector and what properties did we like?}
	
	\begin{table}[H]
		\centering
		\renewcommand\arraystretch{0.5}
		\begin{tabular}{|l|}
			\hline 
			Topics To Write About \\ 
			\hline 
			Monitor vs Projector \\
			DLP VS LED \\  
			1 Projector Per Wall \\ 
			Monitor vs Projector \\ 
			How Did We Choose  \\ 
			Use of a Different Projector for Floor  \\ 
			\hline 
		\end{tabular}
	\end{table}
\end{center}

The display of computer generated content is dominated by projectors and monitors. Projectors use a lamp and lens to project light onto a surface. Monitors utilize a two dimensional array of light cells to present an image to the user.

You can think of these technologies on a loose spectrum, similar to figure 1. On the left end, a large or far image is required, this is ideal for projectors because the resultant image size is a factor of throw distance. As you move closer to the monitors, a close or small image is preferred. This spectrum is loose because you can apply monitor technology that is ideal for a projector (and vice versa), there is an increase in cost and innovation to make it happen.

The ideal CAVE system would be powered by monitors. They are superior in terms of pixel density and resolution, fast response times, physical size, and ease of calibration. A disadvantage is their bezel will cause some issues with rendering and make content like text hard to read. When we increase the pixel density and resolution, there is a increase in power and computing requirements. A cluster of PCs will be required to render and this will drive costs up. Due to this, affordable CAVE systems prefer projector setups.

Although projection based CAVEs suffer from pixel resolution and brightness, we see a savings in computing and cost. It is not necessary for a CAVE to have multiple projectors per wall, although high-end CAVEs may include it to help bridge the gap in performance between the two techniques.

In this project, we decided with using a single projector per wall in order to minimize the complexity and cost of the system. We created a few requirements to filter when we performed a search.

\begin{center}
	\begin{tabular}{|c|c|}
		\hline 
		1 & 1920x1080 Resolution \\ 
		\hline 
		2 & Capable of 3D stereoscopy \\ 
		\hline 
		3 & Ultra Short Throw \\ 
		\hline 
		4 & Less than \$2000 MSRP \\ 
		\hline 
	\end{tabular} 
\end{center}

An EBU Technical Report from 2012 states that 1080p is technically mature and almost all new flat panel displays use FullHD. By 2012, some 4K displays have entered the high-end sector of the market. \cite{ebuHDTVRef} In 2019, 4K flat panels are becoming commonplace and FullHD displays are ubiquitous. Therefore, we make FullHD a requirement for the resolution of the projector.

Reading the history of affordable IPTs, multiple projectors per wall is a comon theme to enable stereoscopic imagery. Each projector would represent an eye and some technique of shuttering is used to block light. Stereoscopic alignment and portability are special concerns for this technique. Multiple projectors are necessary because 

\Section{Screens}\label{sec:hwScreensSection}\\

\filbreak
\noindent\textbf{Screen Material} \\
\begin{center}
	\textcolor{OliveGreen}{Purpose: What was the process of choosing this screen and what properties did we like?}
	
	\begin{table}[H]
		\centering
		\renewcommand\arraystretch{0.5}
		\begin{tabular}{|l|}
			\hline 
			Topics To Write About \\ 
			\hline 
			Fabric vs Panel \\  
			White vs Projector \\
			Folding the Screen \\
			How we found our screen \\
			Wrapping screen around frame edge \\
			Attaching screen to the frame \\ 
			\hline 
		\end{tabular}
	\end{table}
\end{center}


\filbreak
\noindent\textbf{Frame} \\
\begin{center}
	\textcolor{OliveGreen}{Purpose: How did we choose this frame to hold the screens and what factors became important (e.g. accordion, intuitive, velcro)}
	
	\begin{table}[H]
		\centering
		\renewcommand\arraystretch{0.5}
		\begin{tabular}{|l|}
			\hline 
			Topics To Write About \\ 
			\hline 
			Why Independent Frames \\  
			Corner Shape and Rigidy (Wire for corners) \\
			Conference Pop Up Frames \\
			How we  modified the frame \\ 
			\hline 
		\end{tabular}
	\end{table}
\end{center}

\Section{Structure}\label{sec:hwScreensSection}\\
\begin{center}
    \textcolor{OliveGreen}{
        Purpose: Describe the thought behind the structure design and explain some key design (e.g. transportability, motorized, and material)
    }

    \begin{table}[H]
        \centering
        \renewcommand\arraystretch{0.5}
        \begin{tabular}{|l|}
            \hline 
            Topics To Write About \\ 
            \hline 
            Rear Projected vs Front Projected Structures \\  
            General description of our structure\\
            Motorized  \\
            Cable Management  \\
            Packing  \\
            \hline 
        \end{tabular}
    \end{table}
\end{center}

At a minimum, a traditional CAVE system requires a structure to hold the floor projector. High-End CAE systems that utilize multiple displays per wall will typically integrate them into the wall, increasing the complexity of the structure. Some rear projected CAVES will have the projectors on individual pedestals due to the throw distance required.

Structures are also used to hold auxillary peripherals such as the tracking system and speakers.

Since our CAVE is front-projected, all of the projectors are within the bounds of the system. We opted for a single beam design that goes across the screens to support the projectors and peripherals. 


\begin{center}
	\textcolor{blue}{Image of a frame with peripherals}
\end{center}

The structure is designed not only to support but also reduce the complexity of mounting the hardware during setup. 



\filbreak
\begin{table}[H]
	\centering
	\renewcommand\arraystretch{0.5}
	\begin{tabular}{l|l|l}
		\hline 
		System & Input & Quantity \\ 
		\hline 
		Projectors	&  			&  			\\ 
					&  Display 	&  1 Per 	\\ 
					&  Power 	&  1 Per 	\\ 
					&  Network 	&  1 Per 	\\
		Tracking	&  			&  			\\
					&  Sync 	&  1 	 	\\ 	
					&  Power 	&  2 	 	\\	
	\end{tabular} 
		
	\caption{The required inputs into the system} \label{table:requiredInputs}
\end{table}

A difficulty in the structure design was cable management. Table \ref{table:requiredInputs} showcases the required cables for the system. We alleviated this design by multiplexing the network and power cables, by mounting a box a power strip and network switch can be stored.



\Section{Computer}\label{sec:hwComputerSection}\\
\begin{center}
	\textcolor{OliveGreen}{Purpose: Explain why we needed the GTX 1080ti and what other parts were needed (e.g. secondary graphics card). Provide information on specifications of card}
    
    \begin{table}[H]
        \centering
        \renewcommand\arraystretch{0.5}
        \begin{tabular}{|l|}
            \hline 
            Topics To Write About \\ 
            \hline 
            Explanation of Processing Required \\  
            Discussion about Distributed Cluster vs Single PC \\
            How did the 1080TI change our opinion  \\
            Overall reasons for single pc \\
            \hline 
        \end{tabular}
    \end{table}
\end{center}

CAVEs demand a large amount of computing resources in near real-time.  Ideally, each eye should be rendered at 60Hz minimum for each screen. If a screen has more than one display, then each of these will  of course need to be rendered at the same requirements. Originally, this lead to the use of high-end dedicated hardware (e.g. SGI machines) and once PCs were widely available, distributed clusters.

Although dedicated hardware has fallen out of popularity, high end CAVE system require a cluster of computers in order to drive the displays.  Consumer graphics cards continue getting more powerful (largely in part due to the new applications in research and video games) with better general-purpose performance and the ability to support multiple displays. 

The current generation of Nvidia Graphics Cards are capable of running 4 displays in sync. This ability is what allowed the development of the VDen with 1 projector per wall.

This technique is what we researched when building the VDen. There was no room for a cluster, although it would provide greater performance there is an increase in setup and development difficulty. Not to mention that one PC with a single high-end graphics card is cheaper than multiple PCs with multiple lower-end cards.




\Section{Tracking}\label{sec:hwTrackingSection}\\

\begin{center}
	\textcolor{OliveGreen}{Purpose: Explain the reasoning behind the use of SteamVR lighthouse and how we integrated it into the VDen}
    
    \begin{table}[H]
        \centering
        \renewcommand\arraystretch{0.5}
        \begin{tabular}{|l|}
            \hline 
            Topics To Write About \\ 
            \hline 
            Explanation of Leading Tracking Technologies (Optitrack and Lighthouse) \\  
            Why is the lighthouse system popular \\
            Why did we choose the lighthouse  \\
            \hline 
        \end{tabular}
    \end{table}
\end{center}

A tracking system is central to any VR application. It is a core concept to  render a perspective correct image and providing interaction into the virtual world. 

IR Camera tracking techniques are the most popular at this time. In general, these systems work by emitting light and either watching the reflections or having sensors on the tracked object to directly record the light. 

For scientific tracking, systems were designed to produce the highest quality without regards to cost. By using an array of cameras, any angle can be accurately tracked. However, in recent years the computer gaming industry has dominated the market and their desire is for an affordable experience.

In 2016, Valve released the SteamVR lighthouse system which uses an IR emitter and cameras physically located on the tracked object. Although, the size of the tracked area was small, it offered easy setup and a tracking quality that was good enough for the market at a reasonable prices.

This lighthouse system is very popular for consumer VR experiences. They recently released a smaller tracked object that can be placed on things as well as the ability to build your own object. Now researchers can build their own hardware for VR peripherals.

We utilize the light house tracking system because with only two boxes; it can easily coverthe VDen at an affordable price. For head tracking, we attached a small tracker to a hat that the user can wear, making it easy for a user to use the system.


\clearpage
