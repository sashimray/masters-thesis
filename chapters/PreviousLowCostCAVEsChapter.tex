\newcommand{\ns}{not specified}
\newcommand{\checkmark}{\ding{51}}
\newcommand{\cross}{\ding{55}}

\newcolumntype{R}[2]{%
	>{\adjustbox{angle=#1,lap=\width-(#2)}\bgroup}%
	l%
	<{\egroup}%
}
\newcommand*\rot{\multicolumn{1}{R{45}{1em}}}% no optional argument here, please!


\Chapter{Previous Low Cost CAVEs}\label{chapter:affordableCAVEChapter}

\Section[The MiniCAVE: A Voice Controlled IPT Environment]{
	\textit{The MiniCAVE: A Voice Controlled IPT Environment} 
	\newline By Edward Wegman Et Al \hfill 1999
}

\begin{table}[H]
	\centering
	\renewcommand\arraystretch{0.5}
	\begin{tabular}{r|c|c}
		\hline 
		Topic & System &  \\ 
		\hline 
		Cost 				& \textless \$100,000 		& \cross \\ 
		Ease of Development & \ns 						& \cross \\ 
		Transportability 	& \ns  						& \cross \\ 
		Minimum Room Size 	& 6'x6'x6' 					& \checkmark \\ 
		Ease of Setup 		& \ns 						& \cross \\ 
		Computing Power 	& Cluster of PCs 			& \cross \\ 
		Tracking 			& intentionally left out 	& \cross \\  
		Stereoscopy 		& active stereo 			& \checkmark \\ 
		\hline 
	\end{tabular} 

	\caption{Wegman Et Al compared to VDEN Rubric} \label{tab:wegmanRubric}
\end{table}

This paper marks the earliest attempt found to create a more affordable CAVE system. The researchers found that the 200Mhz Pentium Pro Processor (\$3000) was competitive to the SGI Onyx RE2 (\$120,000) when running matrix-oriented mathematics software. With the advancements in computing and projection, they hypothesized that using a PC system could run a CAVE for less than \$100,000. \cite{wegman}

\filbreak
\Section[Immersive Virtual Reality on Commodity Hardware]{
	\textit{Immersive Virtual Reality on Commodity Hardware} 
	\newline By Robert Belleman Et Al \hfill 2001
}

\begin{table}[H]
	\centering
	\renewcommand\arraystretch{0.5}
	\begin{tabular}{r|c|c}
		\hline 
		Topic & System &  \\ 
		\hline 
		Cost 				& \ns 				& \cross \\ 
		Ease of Development & CAVELib 			& \cross \\ 
		Transportability 	& \ns 				& \cross \\ 
		Minimum Room Size 	& \ns 				& \cross \\ 
		Ease of Setup 		& \ns 				& \cross \\ 
		Computing Power 	& Cluster of PCs 	& \cross \\ 
		Tracking 			& electromagnetic 	& \cross \\ 
		Stereoscopy 		& active stereo 	& \checkmark \\ 
		\hline 
	\end{tabular} 
	
	\caption{Robert Belleman Et Al compared to VDEN Rubric} \label{tab:bellemanRubric}
\end{table}

Robert Belleman Et Al describes the work by researchers at SARA and UvA to build a CAVE system based on commercially hard- and software. By utilizing CAVELib and OpenGL|Performer, they were to minimize porting efforts for preexisting applications. This minimization naturally makes life easier for programmers used to these platforms. Finally, they tested the performance by rendering 204,480 triangles in 39,134 triangle strips without texture mapping in active stereo. The SGI CAVE system rendered at 5.5Hz whereas the PC solution rendered at 6.3Hz.

\filbreak
\Section[Implementation of a Low-Cost CAVE System Based on a Networked PC]{
	\textit{Implementation of a Low-Cost CAVE System Based on a Networked PC} 
	\newline By Po-wei Lin Et Al \hfill 2002
}

\begin{table}[H]
	\centering
	\renewcommand\arraystretch{0.5}
	\begin{tabular}{r|c|c}
		\hline 
		Topic & System &  \\ 
		\hline 
		Cost 				& \textless \$100,000 	& \cross \\ 
		Ease of Development & \ns 					& \cross \\ 
		Transportability 	& \ns  					& \cross \\ 
		Minimum Room Size 	& \ns 					& \cross \\ 
		Ease of Setup 		& \ns					& \cross \\ 
		Computing Power 	& Cluster of PCs 		& \cross \\ 
		Tracking 			& electromagnetic 		& \cross \\ 
		Stereoscopy 		& active stereo 		& \checkmark \\ 
		\hline 
	\end{tabular} 
	
	\caption{Po-wei Lin Et Al compared to VDEN Rubric} \label{tab:poweiRubric}
\end{table}

In 1999, the researchers installed the first CAVE system in China based on an SGI machine, almost immediately after they began implementing an alternative using a cluster of PCs. Po-wei Lin Et Al wrote a custom architecture to render on their CAVE using MPI and OpenGL. In the end, their system rendered 60,000 triangles at 18Hz. They further tested the performance of an SGI Onyx2 versus a single PC with a 3DLabs Wildcat 5110-G Graphics card. A test of rendering 1.2 million triangles, the SGI rendered at 2-3Hz whereas the PC at 8-9Hz.


\filbreak
\Section[Low-Cost, Portable, Multi-Wall Virtual Reality]{
	\textit{Low-Cost, Portable, Multi-Wall Virtual Reality} 
	\newline By Samuel Miller Et Al \hfill 2005
}

\begin{table}[H]
	\centering
	\renewcommand\arraystretch{0.5}
	\begin{tabular}{r|c|c}
		\hline 
		Topic & System &  \\ 
		\hline 
		Cost 				& \textless \$100,000 	& \cross \\ 
		Ease of Development & \ns 					& \cross \\ 
		Transportability 	& \ns  					& \cross \\ 
		Minimum Room Size 	& \ns 					& \cross \\ 
		Ease of Setup 		& \ns 					& \cross \\ 
		Computing Power 	& Cluster of PCs 		& \cross \\ 
		Tracking 			& electromagnetic 		& \cross \\ 
		Stereoscopy 		& active stereo 		& \checkmark \\ 
		\hline 
	\end{tabular} 
	
	\caption{Samuel Miller Et Al compared to VDEN Rubric} \label{tab:millerRubric}
\end{table}

This CAVE system developed by researchers in NASA's Applied Sciences DEVELOP Program is focused on affordability and portability. They presented a solution for \textless\$30,000 as opposed to the goal for a CAVE system at \textless\$100,000 in 1999 with the MiniCAVE. Stereoscopy is derived through the use of two LCD projectors and a mechanical shutter. The researchers found a number of issues with 3D glasses and the shutter. The 3D glasses used polarizing lens and were destructive to the polarized light from LCD projectors. The shutter failed to adapt to DLP technologies due to the internal color wheel. DLP projectors spin a color wheel and use a chip of mirrors to compose an image. Active shutter glasses cycle at a specific speed, this should be such that a fully composed image is displayed for the appropriate speed causing issues with calibration. Lastly, the system still relied on mirrors even though a floor projection is intentionally missing. 



\filbreak
\Section[Practical Design and Implementation of a CAVE System]{
	\textit{Practical Design and Implementation of a CAVE System} 
	\newline By Achille Peternier Et Al \hfill 2007
}

\begin{table}[H]
	\centering
	\renewcommand\arraystretch{0.5}
	\begin{tabular}{r|c|c}
		\hline 
		Topic 				& System 				&  \\ 
		\hline 
		Cost 				& \textless \$100,000 	& \cross \\ 
		Ease of Development & \ns 					& \cross \\ 
		Transportability 	& \ns  					& \cross \\ 
		Minimum Room Size 	& \ns 					& \cross \\ 
		Ease of Setup 		& \ns 					& \cross \\ 
		Computing Power 	& Cluster of PCs 		& \cross \\ 
		Tracking 			& electromagnetic 		& \cross \\ 
		Stereoscopy 		& active stereo 		& \checkmark \\ 
		\hline 
	\end{tabular} 
	
	\caption{Robert Belleman Et Al compared to VDEN Rubric} \label{tab:stifdsamuli}
\end{table}

Peternier et Al's CAVE is composed of 3 walls with a floor, a cluster of PCs, and 8 LCD projectors (1 per eye). They adapted a preexisting internal graphics engine. Like Miller's CAVE, they use single screen for the entire system and wire to attach \cite{miller}. They use an optical tracking approach, but they also tested a custom tracking solution using ARToolkit and fiducials. The final system managed a framerate of 25 while rendering 15,000 triangles in stereo and with one light source casting soft shadows.



\filbreak
\Section[A Virtual Reality Installation]{
	\textit{A Virtual Reality Installation}\textcolor{white}{hackhackhack}
	\newline By Francois Sorbier Et Al \hfill 2008
}

\begin{table}[H]
	\centering
	\renewcommand\arraystretch{0.5}
	\begin{tabular}{r|c|c}
		\hline 
		Topic & System &  \\ 
		\hline 
		Cost 				& \textless \$100,000 	& \cross \\ 
		Ease of Development & \ns 					& \cross \\ 
		Transportability 	& \ns  					& \cross \\ 
		Minimum Room Size 	& \ns 					& \cross \\ 
		Ease of Setup 		& \ns 					& \cross \\ 
		Computing Power 	& Cluster of PCs 		& \cross \\ 
		Tracking 			& electromagnetic 		& \cross \\ 
		Stereoscopy 		& active stereo 		& \checkmark \\
		\hline
	\end{tabular} 
	
	\caption{Robert Belleman Et Al compared to VDEN Rubric} \label{tab:stifdsamuli}
\end{table}

Although monoscopic, this CAVE system tested new strategies for screen materials and tracking while maintaining the goal of affordability and transportability. They handmade the projection screens using tracing paper and remark their lastingness. This paper allowed the projectors to be rear projecting. To cut costs, the researchers created their own tracking system. A tracking system has to determine two things: orientation and position. Orientation is solved by placing an electromagnetic compass on the user's head. Position is through a 



\filbreak
\Section[The LAIR: Lightweight Affordable Immersion Room]{
	\textit{The LAIR: Lightweight Affordable Immersion Room} 
	\newline By Barry Denby Et Al \hfill 2009
}

\begin{table}[H]
	\centering
	\renewcommand\arraystretch{0.5}
	\begin{tabular}{r|c|c}
		\hline 
		Topic & System &  \\ 
		\hline 
		Cost 				& \textless \$100,000 	& \cross \\ 
		Ease of Development & \ns 					& \cross \\ 
		Transportability 	& \ns  					& \cross \\ 
		Minimum Room Size 	& \ns 					& \cross \\ 
		Ease of Setup 		& \ns 					& \cross \\ 
		Computing Power 	& Cluster of PCs 		& \cross \\ 
		Tracking 			& electromagnetic 		& \cross \\ 
		Stereoscopy 		& active stereo 		& \checkmark \\
		\hline 
	\end{tabular} 
	
	\caption{Robert Belleman Et Al compared to VDEN Rubric} \label{tab:stifdsamuli}
\end{table}

\filbreak
\Section[Implementing a low-cost CAVE system using CryEngine2]{
	\textit{Implementing a low-cost CAVE system using CryEngine2} 
	\newline By Alex Juarez Et Al \hfill 2010
}

\begin{table}[H]
	\centering
	\renewcommand\arraystretch{0.5}
	\begin{tabular}{r|c|c}
		\hline 
		Topic & System &  \\ 
		\hline 
		Cost 				& \textless \$100,000 	& \cross \\ 
		Ease of Development & \ns 					& \cross \\ 
		Transportability 	& \ns  					& \cross \\ 
		Minimum Room Size 	& \ns 					& \cross \\ 
		Ease of Setup 		& \ns 					& \cross \\ 
		Computing Power 	& Cluster of PCs 		& \cross \\ 
		Tracking 			& electromagnetic 		& \cross \\ 
		Stereoscopy 		& active stereo 		& \checkmark \\
		\hline 
	\end{tabular} 
	
	\caption{Robert Belleman Et Al compared to VDEN Rubric} \label{tab:stifdsamuli}
\end{table}

\filbreak
\Section[An Affordable Surround-Screen Virtual Reality Display]{
	\textit{An Affordable Surround-Screen Virtual Reality Display} 
	\newline By Carolina Cruz-Neira Et Al \hfill 2010
}

\begin{table}[H]
	\centering
	\renewcommand\arraystretch{0.5}
	\begin{tabular}{r|c|c}
		\hline 
		Topic & System &  \\
		\hline 
		Cost 				& \textless \$100,000 	& \cross \\ 
		Ease of Development & \ns 					& \cross \\ 
		Transportability 	& \ns  					& \cross \\ 
		Minimum Room Size 	& \ns 					& \cross \\ 
		Ease of Setup 		& \ns 					& \cross \\ 
		Computing Power 	& Cluster of PCs 		& \cross \\ 
		Tracking 			& electromagnetic 		& \cross \\ 
		Stereoscopy 		& active stereo 		& \checkmark \\
		\hline 
	\end{tabular} 
	
	\caption{Robert Belleman Et Al compared to VDEN Rubric} \label{tab:stifdsamuli}
\end{table}

\filbreak
\Section[Designing a Low Cost Immersive Environment System Twenty Years After the First CAVE]{
	\textit{Designing a Low Cost Immersive Environment System Twenty Years After the First CAVE} 
	\newline By Richard Fowler Et Al \hfill 2012
}

\begin{table}[H]
	\centering
	\renewcommand\arraystretch{0.5}
	\begin{tabular}{r|c|c}
		\hline
		Topic & System &  \\
		\hline
		Cost 				& \textless \$100,000 	& \cross \\ 
		Ease of Development & \ns 					& \cross \\ 
		Transportability 	& \ns  					& \cross \\ 
		Minimum Room Size 	& \ns 					& \cross \\ 
		Ease of Setup 		& \ns 					& \cross \\ 
		Computing Power 	& Cluster of PCs 		& \cross \\ 
		Tracking 			& electromagnetic 		& \cross \\ 
		Stereoscopy 		& active stereo 		& \checkmark \\
		\hline 
	\end{tabular} 
	
	\caption{Robert Belleman Et Al compared to VDEN Rubric} \label{tab:stifdsamuli}
\end{table}

\filbreak
\Section[Discussion]{Discussion}

\begin{table}[H]
	\centering
	\renewcommand\arraystretch{0.5}
	\begin{tabular}{r|c|c|c|c|c|c|c|c|c|c}
		 & \rot{Wegman} & \rot{Belleman} & \rot{Lin} & \rot{Peternier} & \rot{Miller} & \rot{Sorbier} & \rot{Campbell} & \rot{Cruz-Neira} & \rot{Juarez} & \rot{Fowler} \\ 
		\hline
		Cost 				& \checkmark & \cross &  &  &  &  &  &  &  &  \\ 
		Footprint 			& \checkmark & \cross &  &  &  &  &  &  &  &  \\ 
		Resolution 			& \checkmark & \cross &  &  &  &  &  &  &  &  \\ 
		Floor Projection 	& \checkmark & \cross &  &  &  &  &  &  &  &  \\ 
		Development 		& \checkmark & \cross &  &  &  &  &  &  &  &  \\ 
		Setup and Shipping 	& \checkmark & \cross &  &  &  &  &  &  &  &  \\ 
		Computing 			& \checkmark & \cross &  &  &  &  &  &  &  &  \\ 
		\hline
	\end{tabular}

	\caption{All Implementations compared to VDEN Rubric} \label{tab:overallRubric}
\end{table}

fsdafsdafsa
\clearpage
