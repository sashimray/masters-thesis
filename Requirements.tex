\Chapter{Requirements}

\noindent\textbf{Introduction}\\
	To help guide our development, we created a few key requirements. These not only guide but also help differentiate our solution from the previous implementations.

\filbreak
\noindent\textbf{Cost}\\
	The advent of head-mounted displays has brought forth an expectation of low-cost virtual reality. In 2019, PC Magazine  wrote an article on the Best VR Headsets of 2019, the listed price range goes from \$100 -- \$600. \cite{bestHMDs} 
	In 2010, Carolina Cruz-Neira notes that a typical CAVE system of 3 vertical screens and a floor can easily cost over \$750,000. \cite{ccnSurround}. 
	
	In a world where you can get a VR headset for \$200 and build immersive content, it is easy to see why CAVEs have fallen out of favor. We want to help reinstate the CAVE and provide a solution for less than \$40,000.

\filbreak
\noindent\textbf{Footprint and Room Location}\\
	Not only should a CAVE be used in conference and laboratories, but also in the standard office space. We want to showcase immersive content everywhere, so we will target a maximum footprint of 8' x 8' and a height of 10' for the entire system. 
	Lastly, the room should not be required to be on the first floor with large doors to carry parts through. Each part should be lightweight enough to carry and small enough to fit through a standard door. If the room is on a higher floor, then the the loading on the elevator should not be hampered by the size of the parts.

\filbreak
\noindent\textbf{Resolution}\\
	The current demand for display technologies is 4k resolution, making 1920x1080 ubiquitous.

\filbreak
\noindent\textbf{Floor Projection}\\
	While the front projection is the most important as it draws the most visual area, the floor projection is a debatable runner-up. Not does it help to fill your vision vertically, but also the corner formed provides a large amount of immersion.
	
\filbreak
\noindent\textbf{Development}\\
	Designing new media in a intuitive way is the highlight of research and development for a wide range of fields. Game Engines, such as Unity3D and Unreal, have largely taken over the face of the computer graphics industry. These tools make it easier than ever before to build immersive media. We want an SDK that is compatible with a game engine to render into a CAVE.

\filbreak
\noindent\textbf{Setup and Shipping}\\
	The Emerging Analytics Center goes to several conferences per year all across the world. Bringing the VDEN to these events would be great for PR. Therefore, the final design of the system must be able to be setup within one day and has to fit within the back of a van or shipping container.

\filbreak
\noindent\textbf{Computing}\\
	Although distributed computing is still required for high performance CAVE graphics, the capabilities of a graphics card has risen dramatically. Modern cards can handle a system with 4 displays, hinting at the possibility of running a CAVE.

\clearpage
